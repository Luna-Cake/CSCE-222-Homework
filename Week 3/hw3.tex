% Comment lines start with %
% LaTeX commands start with \

\documentclass[12pt]{article}  % This is an article with font size 12-point
\usepackage{mathtools}
\DeclarePairedDelimiter\ceil{\lceil}{\rceil}
\DeclarePairedDelimiter\floor{\lfloor}{\rfloor}

% Packages add features
\usepackage{times}     % font choice
\usepackage{amsmath}   % American Mathematical Association math formatting
\usepackage{amsthm}    % nice formatting of theorems
\usepackage{latexsym}  % provides some more symbols
\usepackage{fullpage}  % uses most of the page (1-inch margins)
\usepackage{algpseudocode}

\setlength{\parskip}{.1in}  % increase the space between paragraphs

\renewcommand{\baselinestretch}{1.1}  % increase the space between lines

% Convenient renaming of symbols for logic formulas
\newcommand{\NOT}{\neg}
\newcommand{\AND}{\wedge}
\newcommand{\OR}{\vee}
\newcommand{\XOR}{\oplus}
\newcommand{\IMPLIES}{\rightarrow}
\newcommand{\IFF}{\leftrightarrow}

% Actual content starts here.
\begin{document}

\begin{center}         % center all the material between begin and end
{\large                % use larger font
CSCE 222-200, Discrete Structures for Computing, Honors \\  % \\ is line break
Fall 2021 \\
Homework 3 \\
Aakash Haran}
\end{center}

% blank line separates paragraphs.  First line of a paragraph is automatically
% indented.  

\rule{6in}{.1pt}       % horizontal line 6 inches long and .1 point high
                    
\noindent              % don't indent
{\bf Instructions:}    % \bf makes text boldface
                       % \em makes text emphasized (italics)

\begin{itemize}        % makes an itemized list
\item The exercises are from the textbook.  MAKE SURE YOU HAVE THE CORRECT
      EDITION!  You are encouraged to work
      extra problems to aid in your learning; remember, the solutions to 
      the odd-numbered problems are in the back of the book.
\item Each exercise is worth 5 points.
\item Grading will be based on correctness, clarity, and whether your
      solution is of the appropriate length.
\item Always justify your answers.
\item Don't forget to acknowledge all sources of assistance on the cover
      sheet, and write up your solutions on your own.
\item {\em Turn in your pdf file on Canvas by 3:00 PM, Wednesday, October 6.}
\end{itemize}

\rule{6in}{.1pt}       % horizontal line 6 inches long and .1 point high

\noindent
{\bf LaTeX hints:}  Read this .tex file for some explanations that are in
the comments.

Math formulas are enclosed in \$ signs, e.g., {\tt \$x + y = z\$}
becomes $x + y = z$.

Logical operators: $\NOT, \AND, \OR, \XOR, \IMPLIES, \IFF$.

Here is a truth table using the ``tabular'' environment:

\begin{center}
\begin{tabular}{|c|c|}  % two columns, both centered (c), 
                        % divided by vertical lines (|)
\hline                  % horizontal line
$p$ & $\NOT p$ \\       % separate column entries with &
\hline
\hline
T & F \\
\hline
F & T \\
\hline
\end{tabular}
\end{center}

{\bf ** Delete the instructions and the LaTeX hints in your solution. **}

\rule{6in}{.1pt}       % horizontal line 6 inches long and .1 point high

%---------------------------------------------------------------------

% \subsection makes a subsection heading; * leaves it unnumbered.
% (Usually subsections are inside sections, but the \section command
% used a font that was larger than I wanted.)
\subsection*{Exercises for Section 2.4 (pp.\ 177--179):}     

\noindent
{\bf 32(c):}
21215

\noindent
{\bf 36):}
The result is $1 - \frac{1}{n+1}$

%--------------------------------------------------------------------

\subsection*{Exercises for Section 2.5 (pp.\ 186--187):}     

{\em Clarification for 2(e) and 4(e):
To exhibit a one-to-one correspondence, it's sufficient
to give a diagram of the pattern; if you provide the actual function
and prove it is a bijection, you can get extra credit.}

\noindent
{\bf 2(e):}
The result is countably infinite. Lining upt the elements of the set like so:
\\
\begin{center}
$1: (2, 1), 2: (3, 1), 3: (2, 2), 4: (3, 2), 5: (2, 3), 6: (3, 3) ...$
\end{center}
We can create a function from each integer that creates an element of the cartesian product.
\begin{center}
$f(n) = (2 + (n + 1) \mod{2}, n - (n + 1) \mod{2})$
\end{center}

The function is one-to-one. $(f(a) = f(b)) \IMPLIES a = b$. If $a - (a + 1) \mod{2} = b - (b + 1) \mod{2}$,
then $a = b$. Even if a and b have the same odd-even parity, they have to be equal.
The function can be proven that it is onto. $\forall x \in S \exists n(f(n) = x)$. 


\noindent
{\bf 4(a):}
The function should follow a pattern of $f(1) = 1, f(2) = 2, f(3) = 4, f(4) = 5, 
f(5) = 7, f(6) = 8$

\noindent
{\bf (A):}
Combining two countable sets, even if they are countably infinite sets, should always 
result in a countable set. 

\noindent
{\bf (B):}
If there exists a one-to-one function from set A to set B and set B to 
set A, then A and B must have the same cardinality. 

\noindent
{\bf 10a:}
A is the set of all real numbers, B is the set of all real numbers as well.
\\
{\bf 10b:}
A is the set of real numbers less than or equal to 1, B is the set of all real numbers less than 1.
\\
{\bf 10c:}
A is the set of real numbers, B is the set of real numbers bigger than 5.

\noindent
{\bf 34a:}
Showing that $f(x) = \frac{2x-1}{2x(1-x)}$ is one-to-one:
\\
The derivative is $\frac{2x^2-2x+1}{2x^2(1-x^2)}$
\\    
The derivate is always positive for x in the domain $(0, 1)$, Hence
the function is always increasing and should be one-to-one.
\\
{\bf 34b:}
$f(x)$ is a function that reaches positive and negtive infinity in the 
domain $0 \leq x \leq 1$ as the denominator approaches 0 when $x$ approaches
0 and 1.

\noindent
{\bf 38:}
Each element in the domain $(0, 1)$ can be written in 
the form $0.d_{1}d_{2}d_{n}$ and has one-to-one corrseponence 
to a function in the set of functions that map a positive integer
to the set of integers 0 to 9. The set of real numbers
in the domain $(0, 1)$ is uncountable and hence the set of functions must
also be uncountable.

%--------------------------------------------------------------------

\subsection*{Exercises for Section 3.1 (pp.\ 213--216):}     

\noindent
{\bf 28:}
\\
\begin{algorithmic}
      \State $arr \gets [a_{1}, a_{2}, ... , a_{n}]$
      \State $left \gets 1, right \gets n$
      \State $m2 \gets 0$

      \While{$left < right$}
            \State $m2 \gets \floor*{\frac{l + r}{2}}$
            \State $m1 \gets \floor*{\frac{l + m2}{2}}$
            \State $m3 \gets \floor*{\frac{m2 + r}{2}}$

            \If{$val == arr[m1]$}
                  \State return m1
            \EndIf
            \If{$val == arr[m2]$}
                  \State return m2
            \EndIf
            \If{$val == arr[m3]$}
                  \State return m3
            \EndIf
            \If{$val < arr[m1]$}
                  \State $right \gets m1$
            \EndIf
            \If{$val < arr[m2]$}
                  \State $right \gets m2$
                  \State $left \gets m1$
            \EndIf
            \If{$val < arr[m3]$}
                  \State $right \gets m3$
                  \State $left \gets m2$
            \EndIf
            \If{$val < arr[right]$}
                  \State $left \gets m3$
            \EndIf
      \EndWhile
      \State return $-1$
      

\end{algorithmic}

%--------------------------------------------------------------------

\subsection*{Exercises for Section 3.2 (pp.\ 228--231):}     

\noindent
{\bf 28a:} $f(x) = 10$ is not $\Omega(x)$ as there is no such C, k pair
where $10 \geq Cx$ for all $x > k$. Because 10 is a constant, $x$ eventually
exceeds it. 
\\
{\bf 28b:} $f(x) = 3x + 7$ is $\Omega(x)$ as the function is greater for
C = 1, and k = 1. Since it is also $O(x)$, the function is $\Theta(x)$.
\\
{\bf 28c:} $f(x) = x^2 + x + 1$ is $\Omega(x)$, as the function's leading
term has a higher exponent. It is actually $\omega(x)$. However, the function
is not $O(x)$, so therefore, it is not $\Theta(x)$.
\\
{\bf 28d:} $f(x) = 5 log x$ is not $\Omega(x)$ as it's a smaller function
since it's a logarithm. Therefore it's not $\Theta(x)$.
\\
{\bf 28e:} $f(x) = \floor*{x}$ is not $\Omega(x)$ as no matter what C and k
values are chosen, $f(x)$ will not be greater than or equal to $x$ since
decimal values would be floored. Hence, it is also not $\Theta(x)$
\\
{\bf 28f:} $f(x) = \floor*{\frac{x}{2}}$ is also not $\Omega(x)$ as no matter what C and k
values are chosen, $f(x)$ will not be greater than or equal to $x$ since
decimal values would be floored. Hence, it is also not $\Theta(x)$

\noindent
{\bf 50:}
The function $f(x) = a_{n}x^n + a_{n-1}x^{n-1} + ... + a_{1}x + a_{0}$ is $O(g(x))$ where 
$g(x) = x^n$.
There exists at least one C, k pair such that $f(x) \leq C*g(x)$ for $x > k$. 
This is intuitive as we can say that $C \geq a_{n}$, and that it should also
account for the remaining $n$ terms in $f(x)$. Since $g(x)$ is function that
increases quicker than a function with a leading exponenet less than $n$, we know
that we can make C and k appropriately high enough such that $g(x) > (f(x) - a_{n}x^n)$.

%--------------------------------------------------------------------

\subsection*{Exercises for Section 3.3 (pp.\ 241--244):}     

\noindent
{\bf 4:} {\em Justify your answer.}\\
$4 log_{2} n + 2$. Each iteration of the while loop has 4 operations, 
two arithmetic and two assignments. It gets executed $log_{2}$ times and
there are two initial assignments.

\noindent
{\bf 14a:}
The loop can be expanded to show all the calculations.
\\
For i = 1, $y = 3*2 + 1$\\
For i = 2, $y = 7*3 + 1$\\
For i = 3, $y = 22*3 + 0$\\
The result is $y = 66$.
\\
{\bf 14b:}
There are n multiplication and n addition operations.

\noindent
{\bf 26:}
There are 4 preliminary assignment operations. The loop will iterate approximately
$log_{4}n$ times. Each iteration does 26 opertions. Each comparison of the loop
is an operation. There are 12 operations to calculate the indices of the
middle values. There are 6 comparison that use 12 operations due to the indexing
Hence the complexity is $4 + 26log_{n}$.

%--------------------------------------------------------------------

\subsection*{Exercises for Section 5.1 (pp.\ 350--354):}     

\noindent
{\bf 4a:}
$P(1)$ is $1^3 = (\frac{1(1 + 1)}{2})^2$ which is true.
{\bf 4b:}
$1^3 = (\frac{1(1 + 1)}{2})^2$ simplifies to $1 = 1$ which is a true statement.
This proved that the base case is true.
\\
{\bf 4c:}
The inductive hypothesis is that $P(k)$ is true. This is the statement
\[ \sum_{i=1}^{k} i^3 = (\frac{k(k + 1)}{2})^2 \]
\\
{\bf 4d:}
It should be shown that $P(k) \IMPLIES P(k+1)$
\\
{\bf 4e:}
The following statement must be shown.
\begin{center}
$1^3 + 2^3 + ... + k^3 + (k+1)^3 = (\frac{(k + 1)(k + 2)}{2})^2$
\end{center}
The inductive hypothesis can be used to replace the first k terms. This 
yields the statement:
\begin{center}
$(\frac{k(k + 1)}{2})^2 + (k + 1)^3 = (\frac{(k + 1)(k + 2)}{2})^2$
\end{center}
We can distribute the squares into their expressions.
\begin{center}
      $\frac{k^2(k + 1)^2}{4} + (k + 1)^3 = \frac{(k + 1)^2(k + 2)^2}{4}$
\end{center}
We then multiply by four and divide by $(k + 1)^2$.
\begin{center}
      $k^2 + 4(k + 1) = (k + 2)^2$
\end{center}
Expanding the both sides reveals that they are equal.
\begin{center}
      $k^2 + 4k + 4 = k^2 + 4k + 4$
\end{center}
Hence, we have proved that $P(k) \IMPLIES P(k+1)$, so the statement
$\forall n P(n)$ is true.
\\
{\bf 4f:}
Mathematical induction relies on the idea that a statement being true
for a value k implies that is true for k + 1, the next integer. If we prove
the statement for a base case such as n = 1, the smallest positive integer and 
that each case implies the next case is true, this effectively proves the statement
for all integers. $((P(1)) \AND (P(k) \IMPLIES P(k + 1)) \IMPLIES \forall n P(n))$
where $n$ is a positive integrer.

\noindent
{\bf 14:}
The goal is to prove the statement \[ \sum_{k=1}^{n} k2^k = (n - 1)2^{n + 1} + 2 \]
Showing that the statement is true for the base case $n = 1$:
\begin{center}
      $1(2^1) = (1 - 1)2^{1 + 1} + 2$ 
\end{center}
Constructing the inductive hypothesis:
\begin{center}
      \[ \sum_{j=1}^{k} j2^j = (k - 1)2^{k + 1} + 2 \]
\end{center}
Attempting to prove $P(k) \IMPLIES P(k+1)$
\begin{center}
      $1*2^1 + 2*2^2 + 3*2^3 + ... + k*2^k + (k + 1) * 2^{k + 1} = (k)*2^{k + 2} + 2$
\end{center}
The inductive hypothesis can be used to replace the first k terms.
\begin{center}
      $(k - 1)2^{k + 1} + 2 + (k + 1) * 2^{k + 1} = (k)*2^{k + 2} + 2$
\end{center}
Subtracting the constants and expanding the left side reveals the follwing statement:
\begin{center}
      $(k)2^{k + 1} + (k) * 2^{k + 1} + 2^{k + 1} - 2^{k + 1}= (k)*2^{k + 2}$
\end{center}
Cancelling terms and simplifying the statements proves their equality:
\begin{center}
      $2*(k)2^{k + 1}= (k)*2^{k + 2}$
\end{center}
Hence, since $P(1)$ is true and $P(k) \IMPLIES P(k+1)$, the statement
$\forall n P(n)$ where $n$ is a positive integer is true. 
\noindent
{\bf 20:}
The goal is to prove the statement \[ \sum_{k=1}^{n} k2^k = (n - 1)2^{n + 1} + 2 \]
Showing that the statement is true for the base case $n = 7$:
\begin{center}
      $3^7 < 7!$ is $2187 < 5040$
\end{center}
Constructing the inductive hypothesis:
\begin{center}
      $3^k < k!$
\end{center}
Attempting to prove $P(k) \IMPLIES P(k + 1)$
\begin{center}
      $3^{k + 1} < (k + 1)!$
\end{center}
We can split the left and right sides and use the inductive hypothesis:
\begin{center}
      $3^k3^1 < (k + 1)(k)!$
\end{center}
Since we know from the inductive hypothesis that $3^k < k!$ and that
$3^1 < k + 1$ since k is an integer greater than 6, the statement is true.
\\
Hence, since $P(1)$ is true and $P(k) \IMPLIES P(k + 1)$, the statement
$\forall n P(n)$ where $n$ is an integer greater than 6 is true.
\noindent
\\
{\bf 40:}
Showing that $(A_{1} \cap A_{2} \cap ... \cap A_{n}) \cup B= (A_{1} \cup B) \cap (A_{2} \cup B) \cap ... \cap (A_{n} \cup B)$ is true for all $n$.
\\
We can see this relationship in a smaller case: $(A \cap B) \cup C$.
If $k \epsilon (A \cap B) \cup C$ then $k$ is an element of $C$ or 
$k$ is an element of $(A \cap B)$ or both. So we can say that $k$ should 
be a member of $A$ or $C$ and $B$ or $C$. This evaluates to $k \epsilon ((A \cap C) \cup (B \cap C))$.
This is a proof of the distributive law and can be applied to any number of sets.
\end{document}
