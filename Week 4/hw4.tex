% Comment lines start with %
% LaTeX commands start with \

\documentclass[12pt]{article}  % This is an article with font size 12-point
\usepackage{mathtools}

% Packages add features
\usepackage{times}     % font choice
\usepackage{amsmath}   % American Mathematical Association math formatting
\usepackage{amsthm}    % nice formatting of theorems
\usepackage{latexsym}  % provides some more symbols
\usepackage{fullpage}  % uses most of the page (1-inch margins)
\usepackage[noend]{algpseudocode}


\setlength{\parskip}{.1in}  % increase the space between paragraphs

\renewcommand{\baselinestretch}{1.1}  % increase the space between lines

% Convenient renaming of symbols for logic formulas
\newcommand{\NOT}{\neg}
\newcommand{\AND}{\wedge}
\newcommand{\OR}{\vee}
\newcommand{\XOR}{\oplus}
\newcommand{\IMPLIES}{\rightarrow}
\newcommand{\IFF}{\leftrightarrow}

% Actual content starts here.
\begin{document}

\begin{center}         % center all the material between begin and end
{\large                % use larger font
CSCE 222-200, Discrete Structures for Computing, Honors \\  % \\ is line break
Fall 2021 \\
Homework 4 \\
Aakash Haran}
\end{center}

% blank line separates paragraphs.  First line of a paragraph is automatically
% indented.  

\rule{6in}{.1pt}       % horizontal line 6 inches long and .1 point high
                    
\noindent              % don't indent
{\bf Instructions:}    % \bf makes text boldface
                       % \em makes text emphasized (italics)

\begin{itemize}        % makes an itemized list
\item The exercises are from the textbook.  MAKE SURE YOU HAVE THE CORRECT
      EDITION!  You are encouraged to work
      extra problems to aid in your learning; remember, the solutions to 
      the odd-numbered problems are in the back of the book.
\item Each exercise is worth 5 points.
\item Grading will be based on correctness, clarity, and whether your
      solution is of the appropriate length.
\item Always justify your answers. 
\item Don't forget to acknowledge all sources of assistance on the cover
      sheet, and write up your solutions on your own.
\item {\em Turn in your pdf file on Canvas by 3:00 PM, Wednesday, October 27.}
\end{itemize}

\rule{6in}{.1pt}       % horizontal line 6 inches long and .1 point high

\noindent
{\bf LaTeX hints:}  Read this .tex file for some explanations that are in
the comments.

Math formulas are enclosed in \$ signs, e.g., {\tt \$x + y = z\$}
becomes $x + y = z$.

Logical operators: $\NOT, \AND, \OR, \XOR, \IMPLIES, \IFF$.

Here is a truth table using the ``tabular'' environment:

\begin{center}
\begin{tabular}{|c|c|}  % two columns, both centered (c), 
                        % divided by vertical lines (|)
\hline                  % horizontal line
$p$ & $\NOT p$ \\       % separate column entries with &
\hline
\hline
T & F \\
\hline
F & T \\
\hline
\end{tabular}
\end{center}

{\bf ** Delete the instructions and the LaTeX hints in your solution. **}

\rule{6in}{.1pt}       % horizontal line 6 inches long and .1 point high

%---------------------------------------------------------------------

% \subsection makes a subsection heading; * leaves it unnumbered.
% (Usually subsections are inside sections, but the \section command
% used a font that was larger than I wanted.)
\subsection*{Exercises for Section 5.2 (pp.\ 362--365):}     

\noindent
{\bf 4a:}
\\
P(18) is true as 2(7) + 1(4) = 18\\
P(19) is true as 1(7) + 3(4) = 19\\
P(20) is true as 0(7) + 5(4) = 20\\
P(21) is true as 3(7) + 0(4) = 21\\
\\
{\bf 4b:} Assume P(18) to P(k) is true.
\\
\\
{\bf 4c:} Prove that P(k + 1) is true with the inductive hypothesis.
\\
\\
{\bf 4d:} k + 1 = (k - 6) + 7. We know we can use a 7 and that P(k - 6) is 
true through the inductive hypothesis.
\\
\\
{\bf 4e:} All cases from 18 to k prove case k + 1. This logic repeats
to prove all cases where $n \geq 18$.

\noindent
{\bf 12:}
Base case:\\ 
$P(1)$ is true. $1 = 2^0$
\\
Inductive Hypothesis:\\
Assume $P(1)$ to $P(k)$ is true.
\\
Proving $P(k + 1)$:\\
If $k + 1$ is even then we know $\frac{k+1}{2}$ is an integer and that
$P(\frac{k + 1}{2})$ is a true statement as it's in the inductive 
hypothesis. If $k + 1$ is odd then we can split it into $k$ and $1$.
We know $k$ is true, and that $1 = 2^0$. Hence, $P(k + 1)$ is true.
$P(n)$ is true for $n \geq 1$.

\noindent
{\bf 32:} The proof is wrong because it assumes $P(k)$ uses 4 cent stamps or 
3 cent stamps which may not necessarily be true.

%---------------------------------------------------------------------

\subsection*{Exercises for Section 5.3 (pp.\ 378--381):}

\noindent
{\bf 12:}
Using strong induction.
\\
Proving $P(1)$:
\\
$P(1)$ says $0^2 = 0 \times 1$, which is a true statement.
\\
Inductive hypothesis:
Assume statements $P(1)$ through $P(k)$ are true.
\\
Attempting to prove $P(k+1)$ which says $\sum\limits_{i=1}^{k+1} f_{i}^2 = f_{k + 1}f_{k+2}$
\\
The left side can be rewritten to make $f_{k}f_{k+1} + f_{k + 1}^2 = f_{k + 1}f_{k+2}$
\\
Factoring $f_{k+1}$ from the left side, we get $f_{k+1}(f_{k} + f_{k + 1}) = f_{k + 1}f_{k+2}$
\\
From the property of the fibonacci sequence we know that $f_{k} + f_{k + 1}$ is 
just $f_{k + 2}$. Hence $P(k + 1)$ is true. Thus the statement 
$\forall n P(n)$ is true when  $n \geq 1$.


\noindent
{\bf 24(c):}
If $x$ and $y$ are in the set, then $x + y$ and $x \times y$ are in the set.
\\
Basis set: $S = (x, 1)$.

\noindent
{\bf 28(c):}
\\
Basis step:
\\
$5 | 0 + 0$ is as true statement as $0$ is divisible by $5$.
\\
Inductive Hypothesis:
\\
We can assume $5 | (a + b)$ is true for all $(a, b) \in S$
\\
Inductive step:
\\
New elements of the set are recursively defined as 
if $(a, b) \in S$ then $(a + 2, b + 3)$ and $(a + 3, b + 2) \in S$.
For the recursive elements, to prove $5 | a + 2 + b + 3$ we can split it into
$5 | (a + b) \wedge 5 | (2 + 3)$. We know the left part is true from the 
inductive hypothesis and that the second part is true since $5$ is divisible by 
$5$. Hence, we proved the statement.

\noindent
{\bf 46:}
Proving that the property is true for the basis step:\\
A tree with one node, the root node fits the property as the number of
leaves is 1, which is 1 more than the number of internal vertices which is 
0. 
\\
Assuming the property is true for full trees $T_{1}$ and $T_{2}$. We can define a new 
full tree $T_{3}$ that is the combination of $T_{1}$ and $T_{2}$ with a new 
root joining the two existing roots. If $T_{1}$ had $n_{1}$ internal vertices 
and $T_{2}$ had $n_{2}$ interal vertices then the new tree has 
$n_{1} + n_{2} + 1$ internval vertices and the number of leaves is the addition 
of the existing leaves which is $n_{1} + 1 + n_{2} + 1$. This value is 1 more 
than the new number of internval vertices hence the property is true for all 
full trees.
%---------------------------------------------------------------------

\subsection*{Exercise Not From the Book:}

\noindent
{\bf (A):}  Prove that the recursive binary search algorithm on slide 69
of Set 7 of the lecture notes is correct.  Use strong induction to show
that $P(n)$ is true for all integers $n \ge 0$, where $P(n)$ is the statement
``Binary-search$(i,j,x)$, where $j-i= n$, returns 0 if $x$ is not in $A[i..j]$
and returns the index of $x$ if $x$ is in $A[i..j]$.''
\\
Base case: 
\\
The recursive algorithm will correctly terminate when there is only one element and when 
there are only two elements. The first case will calculate the middle index 
as 0 and the second will calculate the middle index and adjust to find the 
target element whether it was the index 0 or 1. 
\\
Inductive step:
\\
We can assume that for lengths arrays from 1 to k, the recursive binary search 
algorithm terminates correctly.
\\
Proving the recursive step:
\\
For a length k + 1, the array can be split into (k+1)/2 if k + 1 is even,
and both cases terminate correctly from the inductive hypothesis. If it's not 
even then we have k and an array of size 1, both of which are cases
that work from the inductive hypothesis.
\\
Hence, the recursive binary search algorithm is correct.


%---------------------------------------------------------------------

\subsection*{Exercises for Section 5.4 (pp.\ 391--392):}

\noindent
{\bf 10:}  Just give the algorithm here; the next exercise asks you to
prove its correctness.
\\
\begin{algorithmic}
      \Function{FindMax}{index, arr}:
            \If{$index == len(arr) - 1$}
                  \State return $arr[index]$
            \EndIf
            \State return $max(arr[index], FINDMAX(index + 1, arr))$
      \EndFunction
\end{algorithmic}

\noindent
{\bf 22:}
\\
For an array of length 1, $FINDMAX$ returns the element at index 0, hence
the recursive function is correct for the base case.
\\
Inductive step: we can assume the function works for an array of length $n$.
\\
To prove that $FINDMAX(n+1)$ returns the maximum of $n + 1$ elements, we know that
$FINDMAX(n)$ will return the maximum of $n$ elements from the inductive 
step. The final call of the function returns $max(FINDMAX(n), arr[n])$ which 
is the maximum element of the set. Hence, the recursive function is correct.

\noindent
{\bf 24:} 
\\
\begin{algorithmic}
      \Function{CALCPOWER}{a, n}:
            \If{$n == 1$}
                  \State return $a \times a$
            \EndIf
            \State return $a \times a \times CALCPOWER(a, n - 1)$
      \EndFunction
\end{algorithmic}

\noindent
{\bf 42:} 
** YOUR ANSWER GOES HERE **

%---------------------------------------------------------------------

\subsection*{Exercise for Section 5.5 (p.\ 398):}

\noindent
{\bf 6:} 
\\
Lemma 1 state that every polygon with at least four sides has at least one 
interior diagonal.
\\
\begin{algorithmic}
      \Function{TRIANGULATE}{polygon}:
            \If{polygon has more than 3 sides}
                  \State draw an interior diagonal to split
                  polygon into two smaller polygons, $a$ and $b$.
                  TRIANGULATE($a$), TRIANGULATE($b$)
            \EndIf
      \EndFunction
\end{algorithmic}

\end{document}
