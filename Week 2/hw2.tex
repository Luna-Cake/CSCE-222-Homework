% Comment lines start with %
% LaTeX commands start with \

\documentclass[12pt]{article}  % This is an article with font size 12-point

% Packages add features
\usepackage{times}     % font choice
\usepackage{amsmath}   % American Mathematical Association math formatting
\usepackage{amsthm}    % nice formatting of theorems
\usepackage{latexsym}  % provides some more symbols
\usepackage{fullpage}  % uses most of the page (1-inch margins)
\usepackage{mathtools}
\usepackage{amssymb}

\setlength{\parskip}{.1in}  % increase the space between paragraphs

\renewcommand{\baselinestretch}{1.1}  % increase the space between lines

% Convenient renaming of symbols for logic formulas
\newcommand{\NOT}{\neg}
\newcommand{\AND}{\wedge}
\newcommand{\OR}{\vee}
\newcommand{\XOR}{\oplus}
\newcommand{\IMPLIES}{\rightarrow}
\newcommand{\IFF}{\leftrightarrow}

% Actual content starts here.
\begin{document}

\begin{center}         % center all the material between begin and end
{\large                % use larger font
CSCE 222-200, Discrete Structures for Computing, Honors \\  % \\ is line break
Fall 2021 \\
Homework 2 \\
Aakash Haran}
\end{center}

% blank line separates paragraphs.  First line of a paragraph is automatically
% indented.  

\rule{6in}{.1pt}       % horizontal line 6 inches long and .1 point high
                    
\noindent              % don't indent
{\bf Instructions:}    % \bf makes text boldface
                       % \em makes text emphasized (italics)

\begin{itemize}        % makes an itemized list
\item The exercises are from the textbook.  MAKE SURE YOU HAVE THE CORRECT
      EDITION!  You are encouraged to work
      extra problems to aid in your learning; remember, the solutions to 
      the odd-numbered problems are in the back of the book.
\item Each exercise is worth 5 points.
\item Grading will be based on correctness, clarity, and whether your
      solution is of the appropriate length.
\item Always justify your answers.
\item Don't forget to acknowledge all sources of assistance on the cover
      sheet, and write up your solutions on your own.
\item {\em Turn in your pdf file on Canvas by 3:00 PM on Wednesday, 
      Sep 22.}
\end{itemize}

\rule{6in}{.1pt}       % horizontal line 6 inches long and .1 point high

\noindent
{\bf LaTeX hints:}  Read this .tex file for some explanations that are in
the comments.

Math formulas are enclosed in \$ signs, e.g., {\tt \$x + y = z\$}
becomes $x + y = z$.

Logical operators: $\NOT, \AND, \OR, \XOR, \IMPLIES, \IFF$.

Here is a truth table using the ``tabular'' environment:

\begin{center}
\begin{tabular}{|c|c|}  % two columns, both centered (c), 
                        % divided by vertical lines (|)
\hline                  % horizontal line
$p$ & $\NOT p$ \\       % separate column entries with &
\hline
\hline
T & F \\
\hline
F & T \\
\hline
\end{tabular}
\end{center}

{\bf ** Delete the instructions and the LaTeX hints in your solution. **}

\rule{6in}{.1pt}       % horizontal line 6 inches long and .1 point high

%---------------------------------------------------------------------

% \subsection makes a subsection heading; * leaves it unnumbered.
% (Usually subsections are inside sections, but the \section command
% used a font that was larger than I wanted.)
\subsection*{Exercises for Section 1.6 (pp.\ 82--84):}     

\noindent
{\bf 12:}  (Use the result of Exercise 11 without proving it.  I found it
easier to first use the rules of inference and then use Exercise 11.)

\begin{center}

\begin{tabular}{|c|c|}  % two columns, both centered (c), 
            % divided by vertical lines (|)
\hline                  % horizontal line
$(p \AND t) \IMPLIES (r \OR s)$ & Premise \\       % separate column entries with &
\hline
$q \IMPLIES (u \AND t)$ & Premise \\
\hline
$u \IMPLIES p$ & Premise \\
\hline
$\NOT s$ & Premise \\
\hline
$q$ & Rremise (result from exercise 11)\\
\hline
$u \AND t$ & Modus Ponens\\
\hline
$u$ & Simplification\\
\hline
$t$ & Simplification\\
\hline
$p$ & Modus Ponens\\
\hline
$p \AND t$ & Conjunction\\
\hline
$r \OR s$ & Modus Ponens\\
\hline
$r$ & Disjunctive Syllogism\\
\hline
\end{tabular}

\end{center}

\noindent
{\bf 20(a):}
It is not a true statement. Even if $a^{2}$ is a positive real number the square root
can be a negative number.

\noindent
{\bf 28:}
(Hint:  Work backwards, assuming that you did end up with nine zeros.)
\\
$\NOT R(x) \IMPLIES \forall x(P(x) \OR \NOT Q(x))$, the contrapositive of the initial premise.\\
Since $\forall x(P(x) \OR Q(x))$ is a premise, it has to be that $\forall x (\NOT R(x) \IMPLIES P(x))$. It is implied that the 
value really depends on $P(x)$ since $Q(x)$ appears in both its original and negated form in the statements.


%--------------------------------------------------------------------

\subsection*{Exercises for Section 1.8 (pp.\ 113--117):}     

\noindent
{\bf 6:} 
\begin{center}
      \begin{tabular}{|c|c|c|c|c|}  % two columns, both centered (c), 
                              % divided by vertical lines (|)
      \hline                  % horizontal line
      $a$ & $b$ & $c$ & $min(a, min(b, c))$ & $min(min(a, b), c)$\\       % separate column entries with &
      \hline
      \hline
      1 & 0 & -2 & -2 & -2 \\
      \hline
      -1.4 & -3.5 & -7.4 & -7.4 & -7.4 \\
      \hline
      0 & 0 & 0 & 0 & 0 \\
      \hline
      1.1 & 1.2 & 1.3 & 1.1 & 1.1 \\
      \hline
      \end{tabular}
      \end{center}

\noindent
{\bf 8:}
If $x = 2$ and $y = 3$, we have $5(2) + 5(3) = 25$. Without loss of generality, we do 
not have to test a case where x is odd and y is even as it is identical to this case.

\noindent
{\bf 28:} Proof by contradiction. Assuming we started with 9 0s, then the previous configuration
would've had to be all 1's. In order for that configuration to be possible, its 
previous configuration would have to have alternating 1's and 0's without any 
consecutive element being the same. However, since there are 9 positions, two 
of the same numbers are always consecutive. Hence, it is impossible to end up at 
a configuration of 9 0s.
\\
\noindent
{\bf 36:}
Using a proof by contradiction.\\
Assume $\sqrt[3]{2} = \frac{a}{b}$ where a and b are relatively prime positive integers.\\
It follows that $2b^{3} = a^{3}$ by multiplying both sides by b and cubing them.\\
This equation shows that $a^{3}$ is an even integer, implying that $a = 2c$. \\
The first equation be rewritten with this substitution as $b^{3} = 4c^{3}$. This implies that b
is also even. Hence, since a and b are even integers, we know that they are not relatively
prime, so the cube root of 2 is irrational.

%--------------------------------------------------------------------

\subsection*{Exercises for Section 2.1 (pp.\ 131--133):}     

\noindent
{\bf 10(b)--(f):} Be careful, as $\{2\}$ is NOT the same as $2$.\\
b: 2 is not an element of the set.\\
c: 2 is an element of the set.\\
d: 2 is not an element of the set.\\
e: 2 is not an element of the set.\\
f: 2 is not an element of the set.\\

\noindent
{\bf 26:}
Only subdivision d is a powerset of the set ${a, b}$.

\noindent
{\bf 50:} If S is member of S, this leads to a contradiction. S contains the set 
of sets that do not belong to itself, but in this case S belongs to itself and so 
shouldn't be in the set. However, if S were not in the set S, then it is a set that
doesn't contain itself and should be in the set S, hence it's contradiction either
way. 

%--------------------------------------------------------------------

\subsection*{Exercises for Section 2.2 (pp.\ 144--146):}     

\noindent
{\bf 34:}
\\
Using the provided identity, $(A-B) \cap (B-C) \cap (A-C)$ turns to 
$ (A \cap \widetilde B) \cap (B \cap \widetilde C) \cap (A \cap \widetilde C)$. 
Using the commutative law we get $ (A \cap \widetilde A) \cap (B \cap \widetilde B) \cap (C \cap \widetilde C)$.
The intersection of a set and its negation is the empty set.
Hence, the statement simplifies to the empty set.

\noindent
{\bf 72(e):}  Only consider the special case where $A$ is a subset of $B$
and $B$ is a subset of $C$.
\\
If the cardinality of A, B, and C are 1, 2, and 3 respectively, it is easy to construct
an inequality. This also implies that A is a proper subset of B and B is a proper
subset of C. The inequaltiy that forms is $\frac{2}{3} \le \frac{5}{6}$ which is true.
In the case where cardinality of all three sets is 1 or 0, then we get $0 \le 0$.

%--------------------------------------------------------------------

\subsection*{Exercises for Section 2.3 (pp.\ 161--164):}     

\noindent
{\bf 34 a:} Every value being able to map from A to C implies 
that for every value c in C, there is a b in B such that 
f(b) = c. b is derived from a value x in A, such that g(x) = b. Therefore, for every
element c in C, there is an x in A, such that f(g(x)) = c implying that f is onto.
\\
{\bf 34 b:} If f(g(x)) is one-to-one this means that f(a) = f(b) implies a = b. 
For a and b, there are values in A, x and y, such that g(x) = a and g(y) = b. Since
a = b, g(x) = g(y) when g is one-to-one. This means x = y, and so if f(g(x)) is one-to-one, g(x) is one-to-one.
\\
{\bf 34 c:} If f(g(x)) is a bijection, then for every c in C, there is a x in A such 
that f(g(x)) = c and if f(g(a)) = f(g(b)) then g(a) = g(b). If g(x) is onto, this 
means for every b in B, there is an x in A such that g(x) = b. This means if f(g(x))
is a bijection then f(a) = f(b) implies a = b. a and b are elements in B. There is 
an x and y in A such that g(x) = a, and g(y) = b. Hence, g(x) = g(y). Therefore, if 
g is onto then f is one-to-one.


Assuming f is one-to-one, this means we know that f(g(x)) = f(b) implies g(x) = b.
Hence, since f(g(x)) is a bijection, then for every element b in B there must be a 
x in A such that g(x) = a. Hence, the statement that if f(g(x)) is a bijection
then f is one-to-one iff g is onto is true.

\noindent
{\bf 76(e):}
This is a false statement. Suppose $x = 0.5$ and $y = 0.5$, then 
$\lfloor 0.5 \rfloor + \lfloor 0.5 \rfloor + \lfloor 1 \rfloor = \lfloor 2(0.5) \rfloor + 
\lfloor 2(0.5) \rfloor$. This leads to the statement $1 = 2$ which is false.

\end{document}
