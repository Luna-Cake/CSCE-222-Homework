% Comment lines start with %
% LaTeX commands start with \

\documentclass[12pt]{article}  % This is an article with font size 12-point

% Packages add features
\usepackage{mathtools}
\usepackage{times}     % font choice
\usepackage{amsmath}   % American Mathematical Association math formatting
\usepackage{amsthm}    % nice formatting of theorems
\usepackage{latexsym}  % provides some more symbols
\usepackage{fullpage}  % uses most of the page (1-inch margins)
\usepackage{tikz}
\usetikzlibrary{arrows}

\DeclarePairedDelimiter\ceil{\lceil}{\rceil}
\DeclarePairedDelimiter\floor{\lfloor}{\rfloor}

\setlength{\parskip}{.1in}  % increase the space between paragraphs

\renewcommand{\baselinestretch}{1.1}  % increase the space between lines

% Convenient renaming of symbols for logic formulas
\newcommand{\NOT}{\neg}
\newcommand{\AND}{\wedge}
\newcommand{\OR}{\vee}
\newcommand{\XOR}{\oplus}
\newcommand{\IMPLIES}{\rightarrow}
\newcommand{\IFF}{\leftrightarrow}

% Actual content starts here.
\begin{document}

\begin{center}         % center all the material between begin and end
{\large                % use larger font
CSCE 222-200, Discrete Structures for Computing, Honors \\  % \\ is line break
Fall 2021 \\
Homework 5 \\
Aakash Haran}
\end{center}

% blank line separates paragraphs.  First line of a paragraph is automatically
% indented.  

\rule{6in}{.1pt}       % horizontal line 6 inches long and .1 point high
                    
\noindent              % don't indent
{\bf Instructions:}    % \bf makes text boldface
                       % \em makes text emphasized (italics)

\begin{itemize}        % makes an itemized list
\item The exercises are from the textbook.  MAKE SURE YOU HAVE THE CORRECT
      EDITION!  You are encouraged to work
      extra problems to aid in your learning; remember, the solutions to 
      the odd-numbered problems are in the back of the book.
\item Each exercise is worth 5 points.
\item Grading will be based on correctness, clarity, and whether your
      solution is of the appropriate length.
\item Always justify your answers. 
\item Don't forget to acknowledge all sources of assistance on the cover
      sheet, and write up your solutions on your own.
\item {\em Turn in your pdf file on Canvas by 3:00 PM, Wednesday, November 10.}
\end{itemize}

\rule{6in}{.1pt}       % horizontal line 6 inches long and .1 point high

\noindent
{\bf LaTeX hints:}  Read this .tex file for some explanations that are in
the comments.

Math formulas are enclosed in \$ signs, e.g., {\tt \$x + y = z\$}
becomes $x + y = z$.

Logical operators: $\NOT, \AND, \OR, \XOR, \IMPLIES, \IFF$.

Here is a truth table using the ``tabular'' environment:

\begin{center}
\begin{tabular}{|c|c|}  % two columns, both centered (c), 
                        % divided by vertical lines (|)
\hline                  % horizontal line
$p$ & $\NOT p$ \\       % separate column entries with &
\hline
\hline
T & F \\
\hline
F & T \\
\hline
\end{tabular}
\end{center}

{\bf ** Delete the instructions and the LaTeX hints in your solution. **}

\rule{6in}{.1pt}       % horizontal line 6 inches long and .1 point high

%---------------------------------------------------------------------

% \subsection makes a subsection heading; * leaves it unnumbered.
% (Usually subsections are inside sections, but the \section command
% used a font that was larger than I wanted.)

\subsection*{Exercises for Section 6.1 (pp.\ 416--420):}     

\noindent
{\bf 22 (b), (c), (d), (e), (f):} 
\\
130, 12, 220, 208, 780

\noindent
{\bf 32 (a), (b), (c), (d), (e):
\\
$26^8$, $26 \times 25 \times 24 \times 23 \times 22 \times 21 \times 20 \times 19$, 
$26 \times 25 \times 24 \times 23 \times 22 \times 21 \times 20$, 
$25 \times 24 \times 23 \times 22 \times 21 \times 20 \times 19$.

\rule{6in}{.1pt}       % horizontal line 6 inches long and .1 point high

%---------------------------------------------------------------------

\subsection*{Exercises for Section 6.2 (pp.\ 426--428):}     

\noindent
{\bf 4 (a) and (b):}
\\
$5, 13$

\noindent
{\bf 12:}
** YOUR ANSWER GOES HERE **

\rule{6in}{.1pt}       % horizontal line 6 inches long and .1 point high

%---------------------------------------------------------------------

\subsection*{Exercises for Section 6.3 (pp.\ 435--437):}     

\noindent
{\bf 10: $6!$

\noindent
{\bf 12a:} $C(12, 3)$ 
\\
{\bf 12b:} $C(12, 0) + C(12, 1) + C(12, 2) + C(12, 3)$
\\
{\bf 12c:}
\\
\[ \sum_{n=4}^{12} C(12, n) \]
\\
{\bf 12d:} $C(12, 6)$

\noindent
{\bf 34 (a), (b), (c) and (d):  For (a), interpret as ``at least one $a$'';
for (b), interpret as ``at least one $a$ and at least one $b$''.}
\\
{\bf 34a:} $C(6, 1) \times 26^5$
\\
{\bf 34b:} $C(6, 1) \times C(5, 1) \times 26^4$
\\
{\bf 34c:} $C(5, 1) \times 24 \times 23 \times 22 \times 21$
\\
{\bf 34d:} $15 \times 24 \times 23 \times 22 \times 21$


\rule{6in}{.1pt}       % horizontal line 6 inches long and .1 point high

%---------------------------------------------------------------------

\subsection*{Exercises for Section 6.4 (pp.\ 443--445):}     

\noindent
{\bf 6:} 
$330$

\noindent
{\bf 14:} 
$C(100, \frac{k}{2})$

\noindent
{\bf 18:}
{\normalfont
Each term in the sequence can be written by $a_{n} = a_{n-1} \times \frac{n - k}{k + 1}$ and 
the value of $k$ increases by 1 for each next term. While $k < \frac{n}{2}$ the value of 
$\frac{n - k}{k + 1}$ is greater than 1, so each next term is increasing and when $k = \frac{1}{2}$, the values
of the term and the previous is equal. When $k > \frac{1}{2}$ the value of the multiplicative
constant is less than 1 and the sequence starts decreasing.
}
\\
\noindent
{\bf 20a:}
{\normalfont 
Using the result from exercise 18, we know that the largest term in the 
sequence is $C(n, \floor*{n/2})$ and that there are $n$ terms. The sum of 
all the terms is upper bounded by $sum = C(n, \floor*{n/2}) \times n$. We 
can then say that $\frac{sum}{n} \leq C(n, \floor*{n/2})$.
}
\\
{\bf 20b:}
{\normalfont 
We can adjust the result from question 18, $\sum_{k=0}^{2n} C(2n, k) = 2^{2n} = 4^n$.
The upper bound for the sum of the terms is $sum = C(2n, n) \times n$. 
Thus, $\frac{sum}{2n} \leq C(2n, n)$.
}

\noindent
{\bf 36:} 
\\
{\normalfont
Attempting to prove the statement for all values of n through 
regular induction. 
\\
Statement: $P(n)$ says $(x+y)^n = \sum_{k=0}^{n} C(n, k)x^{n-k}y^k$
\\
Base case: $(x+y)^0 = C(0, 0)x^{0-0}y^0 = 1$. The statement is true for the
base case.
\\
Inductive Step: Assume $P(0)$ to $P(n)$ is true.
\\
Attempting to prove $P(n+1)$ which is $(x+y)^{n+1} = \sum_{k=0}^{n+1} C(n+1, k)x^{n+1-k}y^k$
\\
$(x+y)^{n+1}$ can be expressed as $(x+y)^n(x+y)$. $(x+y)^n$ follows the 
binomial expansion properly from the inductive hypothesis.
\\
Looking at the first few terms in the expansion $C(n, 0)x^{n} + C(n, 1)x^{n-1}y^1 + C(n, 2)x^{n-2}y^2 + \cdots$
it becomes to easy that in each term after the first, multiplying that term by $x$ and 
the previous term by $y$ ensures that they are capable of being added. Multiplying 
$(x+y)^n$ by $x+y$ does exactly this and makes each consecutive term capable of being 
added together. Using the property of Pascal's Triangle that consecutive elements added 
create a new row following the pattern of binomial expansion, we know that that 
is exactly what's happening in this expansion.
\\
Hence, the statement $P(n)$ is true for all $n$.
}

%---------------------------------------------------------------------

\subsection*{Exercises for Section 6.5 (pp.\ 454--457):}     

\noindent
{\bf 10 a:} $6^{12}$
\\
{\bf 10 b:} $6^{36}$
\\
{\bf 10 c:}

\noindent
{\bf 18:} $20! \div 2! \div 4! \div 3! \div 2! \div 3! \div 2! \div 3!$

\noindent
{\bf 24:} $C(17, 12)$

\noindent
{\bf 26:} $C(15, 5) \times C(10, 4) \times C(6, 3) \times C(3, 2)$

\noindent
{\bf 48:} $7! \times 2!$

\rule{6in}{.1pt}       % horizontal line 6 inches long and .1 point high

%---------------------------------------------------------------------

\subsection*{Exercises for Section 9.1 (pp.\ 608--610):}     

\noindent
{\bf 6d:} {\normalfont Vacuously antisymmetric.}
\\
{\bf 6e:} {\normalfont Symmetric, reflexive, transitive}
\\
{\bf 6f:} {\normalfont Symmetric, transitive}
\\
{\bf 6h} {\normalfont Symmetric, transitive}


\noindent
{\bf 30a:} {\normalfont \{(1, 2), (2, 3), (3, 4), (1, 1), (2, 1), (2, 2), (3, 1), (3, 2), (3, 3)\}}
\\
{\bf 30b:} {\normalfont \{(1, 2), (2, 3), (3, 4)\}}

\noindent
{\bf 32:} {\normalfont \{(1, 1), (1, 2), (2, 1), (2, 2)\}}

\noindent
{\bf 52b:} {\normalfont Both relations should have (a, a) for every a in A. Therefore
they are included in the intersection. }
\\
{\bf 52e:} {\normalfont If (a, a) is in R then (a, a) is in S and hence 
the composite will contain (a, a). }

\rule{6in}{.1pt}       % horizontal line 6 inches long and .1 point high

%---------------------------------------------------------------------

\subsection*{Exercises for Section 9.2 (pp.\ 619--621):}     

\noindent
{\bf 8a:} {\normalfont ISBN}
\\
{\bf 8b:} {\normalfont If the title and publication date pair for every book
is unique.}
\\
{\bf 8c:} {\normalfont If the title and number of pages pair for every book
is unique.}
\\
\noindent
{\bf 12}
** YOUR ANSWER GOES HERE **

\noindent
{\bf 16}
** YOUR ANSWER GOES HERE **

\rule{6in}{.1pt}       % horizontal line 6 inches long and .1 point high

%---------------------------------------------------------------------

\subsection*{Exercises for Section 9.3 (pp.\ 629--627):}     

\noindent
{\bf 4 (a)} {\normalfont \{(1, 1), (3, 3), (4, 4), (1, 2), (2, 1), (1, 4), (4, 1), (2, 3), (3, 2), (4, 3), (3, 4) \}}

\noindent
\\
\\
{\bf 14a:} 
\begin{center}
      \begin{tabular}{|c|c|c|}
      \hline
      0 & 1 & 0 \\
      \hline
      1 & 1 & 1 \\
      \hline
      1 & 1 & 1 \\
      \hline
      \end{tabular}
\end{center}
{\bf 14b}
\begin{center}
      \begin{tabular}{|c|c|c|}
      \hline
      0 & 1 & 0 \\
      \hline
      0 & 1 & 1 \\
      \hline
      1 & 0 & 0 \\
      \hline
      \end{tabular}
\end{center}
{\bf 14c}
\begin{center}
      \begin{tabular}{|c|c|c|}
      \hline
      0 & 1 & 1 \\
      \hline
      1 & 1 & 1 \\
      \hline
      0 & 1 & 0 \\
      \hline
      \end{tabular}
\end{center}
{\bf 14d}
\begin{center}
      \begin{tabular}{|c|c|c|}
      \hline
      1 & 1 & 1 \\
      \hline
      1 & 1 & 1 \\
      \hline
      0 & 1 & 0 \\
      \hline
      \end{tabular}
\end{center}

\noindent
{\bf 22}
\\
\begin{tikzpicture}

      \tikzset{vertex/.style = {shape=circle,draw,minimum size=1.5em}}
      \tikzset{edge/.style = {->,> = latex'}}
      % vertices
      \node[vertex] (a) at  (0,0) {$a$};
      \node[vertex] (b) at  (4,3) {$b$};
      \node[vertex] (c) at  (8,0) {$c$};
      \node[vertex] (d) at  (4,-3) {$d$};
      %edges
      \draw[edge] (a) to (a);
      \draw[edge] (a) to (b);
      \draw[edge] (b) to (c);
      \draw[edge] (c) to (b);
      \draw[edge] (c) to (d);
      \draw[edge] (d) to (a);
      \draw[edge] (d) to (b);
      \end{tikzpicture}

\noindent
{\bf 26} {\normalfont \{(a, a), (b, b), (c, c), (d, d), (a, b), (b, a), (c, a), (c, d)\}}

\noindent
{\bf 32 (only for the graph in 26 and only for reflexive, symmetric,
antisymmetric, and transitive)}
\\
{\normalfont
The graph represents reflexive relation.
}

\rule{6in}{.1pt}       % horizontal line 6 inches long and .1 point high

%---------------------------------------------------------------------

\subsection*{Exercises for Section 9.4 (pp.\ 637--638):}     

\noindent
{\bf 6}
\\
\begin{tikzpicture}

      \tikzset{vertex/.style = {shape=circle,draw,minimum size=1.5em}}
      \tikzset{edge/.style = {->,> = latex'}}
      % vertices
      \node[vertex] (a) at  (0,0) {$a$};
      \node[vertex] (b) at  (4,3) {$b$};
      \node[vertex] (c) at  (8,0) {$c$};
      %edges
      \draw[edge] (a) to (b);
      \draw[edge] (b) to (c);
      \draw[edge] (c) to (a);
      \draw[edge] (a) to (a);
      \draw[edge] (b) to (b);
      \draw[edge] (c) to (c);
      \end{tikzpicture}

\noindent
{\bf 22}
** YOUR ANSWER GOES HERE **

\noindent
{\bf 26 (c)}
** YOUR ANSWER GOES HERE **

\rule{6in}{.1pt}       % horizontal line 6 inches long and .1 point high

\end{document}
